\documentclass[11 pt]{article}
% Set target color model to RGB
\usepackage[inner=2.0cm,outer=2.0cm,top=2.5cm,bottom=2.5cm]{geometry}
\usepackage{setspace}
\usepackage[rgb]{xcolor}
\usepackage{verbatim}
\usepackage{subcaption}
\usepackage{amsgen,amsmath,amstext,amsbsy,amsopn,tikz,amssymb,tkz-linknodes}
\usepackage{fancyhdr}
\usepackage[colorlinks=true, urlcolor=blue,  linkcolor=blue, citecolor=blue]{hyperref}
\usepackage[colorinlistoftodos]{todonotes}
\usepackage{rotating}
\usepackage{booktabs}
\usepackage{listings}



\title{PLTL Exercise 3 (Inheritance)}
\author{CCOM 3034 – Data Structures \& Algorithms}
\date{Week 3}

\begin{document}

\maketitle



\section{Instructions}

\textbf{Let's practice inheritance!} \\

\noindent In this exercise you will practice: (1) creating a derived class from a base class, (2) implementing functions for the derived class, and (3) simple string algorithms. \\

\noindent The purpose of this exercise is to create a class called \texttt{SuperString} which is derived from the C++ \verb|string| class. Since the \verb|string| class supports many constructors, we will only be overloading the following:

\begin{itemize}
    \item the default constructor: just leave the body of the \verb|SuperString()| constructor empty and let the C++ inheritance call the \verb|string| by default.
    \item the constructor that receives a \texttt{const string \&st}. Again, leave the body of the \verb|SuperString| \verb|(const string &st)| constructor empty, but call the \verb|string(st)| constructor in the initializer list.
\end{itemize}

\noindent The \texttt{SuperString} class will also support the following member functions:

\begin{itemize}
    \item \texttt{void SuperString::reverse()} - which reverses the \texttt{SuperString}
    \item \texttt{int SuperString::count(char c)} - which counts the number of times that character \texttt{c} appears in the string
    \item \texttt{SuperString SuperString::upperCase() const} - which returns a copy of the original \texttt{SuperString} in uppercase
    \item \verb|SuperString intersection(const SuperString &st)| which returns the letters which appear in both the invoking and the argument string. For example, the intersection of \verb|"mouse"| and \verb|"spam"| is \verb|"sm"| or \verb|"ms"|. Try to make it a $O(n+m)$ algorithm, where $n$ and $m$ are the length of the strings. 
\end{itemize}

\section{Notes}

\noindent Notice that we are not overloading the assignment operator (\texttt{=}). Thus, in your program you won't be able to assign to \texttt{SuperString}s like this:

\begin{center}
    \texttt{SuperString ss = "Hola";}
\end{center}



\end{document}