\documentclass[11 pt]{article}
% Set target color model to RGB
\usepackage[inner=2.0cm,outer=2.0cm,top=2.5cm,bottom=2.5cm]{geometry}
\usepackage{setspace}
\usepackage[rgb]{xcolor}
\usepackage{verbatim}
\usepackage{subcaption}
\usepackage{amsgen,amsmath,amstext,amsbsy,amsopn,tikz,amssymb,tkz-linknodes}
\usepackage{fancyhdr}
\usepackage[colorinlistoftodos]{todonotes}
\usepackage{rotating}
\usepackage{booktabs}
\usepackage{listings}
\lstset{%frame=tb,
  %language=cpp,
  aboveskip=3mm,
  belowskip=3mm,
  showstringspaces=false,
  columns=flexible,
  basicstyle={\ttfamily},
  numbers=none,
  numberstyle=\tiny\color{gray},
  keywordstyle=\color{blue},
  commentstyle=\color{dkgreen},
  stringstyle=\color{mauve},
  breaklines=false,
  breakatwhitespace=false,
  tabsize=3
}


\title{PLTL Exercise 8 (Recursive Time Complexities)}
\author{CCOM 3034 – Data Structures \& Algorithms}
\date{Week 8}

\begin{document}

%\begin{center}
    %\textbf{Universidad de Puerto Rico, R{\'\i}o Piedras}\\
    %Facultad de Ciencias Naturales\\
    %Departamento de Ciencias de C\'omputos\\
    %San Juan, Puerto Rico
%\end{center}

\maketitle

\section{Problems}

Using the Recursion Trees Method, calculate the time complexities of the following recursive functions.

\subsection{Problem 1}

\[ 
    T(N) = 
    \begin{cases} 
        a & N = 0 \\
        b + T(N-1) & N > 0
    \end{cases}
\]

\subsection{Problem 2}

\[ 
    T(N) = 
    \begin{cases} 
        a & N = 1 \\
        b + T(N/2) & N > 1
    \end{cases}
\]

\subsection{Problem 3}

\[ 
    T(N) = 
    \begin{cases} 
        a & N = 0 \\
        b + cN + T(N-1) & N > 0
    \end{cases}
\]

\subsection{Problem 4 \normalfont{(recall this is the complexity of the function \texttt{maxRec} we discussed in class!)}}

\[ 
    T(N) = 
    \begin{cases} 
        a & N = 1 \\
        b + 2T(N/2) & N > 1
    \end{cases}
\]

\subsection{Problem 5}

\[ 
    T(N) = 
    \begin{cases} 
        a & N = 0 \\
        b + 2T(N-1) & N > 0
    \end{cases}
\]

\subsection{Problem 6}

\noindent Notice some functions had $N-1$ as arguments in the recursive step, while others had $N/2$. Will a function that has $N/2$ as argument \textit{always} have a smaller time complexity than one whose argument is $N-1$? Why so (or why not)? Give an example.

\end{document}
