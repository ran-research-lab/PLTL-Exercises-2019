\documentclass[11 pt]{article}
% Set target color model to RGB
\usepackage[inner=2.0cm,outer=2.0cm,top=2.5cm,bottom=2.5cm]{geometry}
\usepackage{setspace}
\usepackage[rgb]{xcolor}
\usepackage{verbatim}
\usepackage{subcaption}
\usepackage{amsgen,amsmath,amstext,amsbsy,amsopn,tikz,amssymb,tkz-linknodes}
\usepackage{fancyhdr}
\usepackage[colorlinks=true, urlcolor=blue,  linkcolor=blue, citecolor=blue]{hyperref}
\usepackage[colorinlistoftodos]{todonotes}
\usepackage{rotating}
\usepackage{booktabs}
\usepackage{listings}



\title{PLTL Exercise 5 (Linked Lists)}
\author{CCOM 3034 – Data Structures \& Algorithms}
\date{Week 5}

\begin{document}

\maketitle



\section{Some Questions on Linked Lists}

In class, we implemented a List ADT using a singly-linked list with a head pointer and a tail pointer, as well as a length data member (for a quick reminder, see its definition below). 

\begin{enumerate}
    \item What is the Big-Oh of the functions \texttt{append}, \texttt{prepend}, \texttt{insertAfter}, and \texttt{search}?
    \item If we didn't have the tail pointer, what would've been the Big-Oh of \texttt{append}? What about the Big-Oh of the other functions? Would they be affected?
    \item If \texttt{insertAfter} didn't receive the parameter \texttt{p}, what would've been its Big-Oh?
    \item Implement a new member function, called \texttt{average}, that returns the average of the whole list. It should be $O(1)$ (Hint: What can we change in the class's definition to help us make it $O(1)$?).
\end{enumerate} \\

\section {\texttt{Node} \& \texttt{Llist} Class Definitions}

%Use the following sample declaration/definition of the \texttt{Llist} class so that you may use its data members and member functions accordingly (should they be necessary):

\begin{lstlisting}[language=C++]
    class Node {
        public:
            int data;
            Node *next;
            Node(int d = 0, Node *n = NULL): data(d), next(n) {}
    };

    class Llist {
        private:
            Node *head, *tail;
            int length;
        public:
            Llist() : head(NULL), tail(NULL), length(0) {}
            void append(int d);
            void prepend(int d);
            void insertAfter(Node *p, int d);
            Node *search(int d) const;
            friend ostream& operator<< (ostream &out, const Llist& L);
    };
    
    
    void Llist::append(int d){
        Node *n = new Node(d);
        if(!head) head = n;
        else tail->next = n;
        tail = n;
        length++;
    }
    
    void Llist::prepend(int d) {
        Node *n = new Node(d, head);
        head = n;
        if(length == 0) tail = n;
        length++;
    }
    
    void Llist::insertAfter(Node *p, int d) {
        Node *n;
        if(p == NULL) {
            n = new Node(d, head);
            head = n;
            if(!tail) tail = n;
        }
        else {
            n = new Node(d, p->next);
            p->next = n;
            if(tail == p) tail = n;
        }
        length++;
    }
    
    Node* Llist::search(int d) const {
        Node *p = head;
        while(p) {
            if(p->data == d) return p;
            p = p->next;
        }
        return NULL;
    }

    ostream& operator<< (ostream &out, const Llist& L) {
        Node *p = L.head;
        while(p) {
            out << p->data << " ";
            p = p->next;
        }
        return out;
    }
\end{lstlisting}

%\section{Notes}

%Since this exercise uses dynamic memory allocation, you are urged to use the \texttt{delete} operator to avoid memory leaks whenever possible. Remember: for each \texttt{new}, there should be a corresponding \texttt{delete}.
%you are \textbf{\underline{strongly urged}} to use an online IDE (Integrated Development Environment) such as \textbf{repl.it} so that no harm is done to your computer for compiling and running faulty code (i.e. when debugging, etc.).

\end{document}
