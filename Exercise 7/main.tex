\documentclass[11 pt]{article}
% Set target color model to RGB
\usepackage[inner=2.0cm,outer=2.0cm,top=2.5cm,bottom=2.5cm]{geometry}
\usepackage{setspace}
\usepackage[rgb]{xcolor}
\usepackage{verbatim}
\usepackage{subcaption}
\usepackage{amsgen,amsmath,amstext,amsbsy,amsopn,tikz,amssymb,tkz-linknodes}
\usepackage{fancyhdr}
\usepackage[colorinlistoftodos]{todonotes}
\usepackage{rotating}
\usepackage{booktabs}
\usepackage{listings}
\lstset{%frame=tb,
  %language=cpp,
  aboveskip=3mm,
  belowskip=3mm,
  showstringspaces=false,
  columns=flexible,
  basicstyle={\ttfamily},
  numbers=none,
  numberstyle=\tiny\color{gray},
  keywordstyle=\color{blue},
  commentstyle=\color{dkgreen},
  stringstyle=\color{mauve},
  breaklines=false,
  breakatwhitespace=false,
  tabsize=3
}


\title{PLTL Exercise 7 (International Recursion Week)}
\author{CCOM 3034 – Data Structures \& Algorithms}
\date{Week 7}

\begin{document}

%\begin{center}
    %\textbf{Universidad de Puerto Rico, R{\'\i}o Piedras}\\
    %Facultad de Ciencias Naturales\\
    %Departamento de Ciencias de C\'omputos\\
    %San Juan, Puerto Rico
%\end{center}

\maketitle

\section{Problems}


\subsection{Problem 1}

Consider the following function \texttt{void skyIsTheLimit(int n)} to answer the questions:
\begin{center}
\begin{tabular}{c}
\begin{lstlisting}
void skyIsTheLimit(int n) {
    if (n == 100) return;
    cout << n << endl;
    skyIsTheLimit(n + 1);
}

\end{lstlisting}
\end{tabular}
\end{center}

\begin{itemize}
    \item Is \texttt{void skyIsTheLimit(int n)} recursive? If so, why?
    \vspace{-.25cm}  
    \item For which values of \texttt n does it work without segmentation fault?
    \vspace{-.25cm}  
    \item What is the output?
\end{itemize}

\subsection{Problem 2}
The function \texttt{int maxRec(int A[], int low, int high)} uses recursion to return the max value of an array. Use hand tracing for an 8 element array.

\begin{center}
\begin{tabular}{c}
\begin{lstlisting}
int maxRec(int A[], int low, int high) {
    if (low == high) return A[low];
    int mid = (high + low) / 2;
    return max(maxRec(A, low, mid), maxRec(A, mid + 1, high));
}
\end{lstlisting}
\end{tabular}
\end{center}

\subsection{Problem 3: \textit {Be a recursion hero!}}
Use your hand tracing powers to determine the value of \texttt{aa(2,2)}. (Hint: \textit {even for such small values, this function is diabolically recursive.}

\begin{center}
\begin{tabular}{c}
\begin{lstlisting}
int aa(int m, int n) {
    if (m == 0)             return n + 1;
    if (m > 0 && n == 0)    return aa(m - 1, 1);
    if (m > 0 && n > 0)     return aa(m - 1, aa(m,n - 1));
}
\end{lstlisting}
\end{tabular}
\end{center}


\end{document}
